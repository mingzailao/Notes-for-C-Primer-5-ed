% Created 2016-09-21 Wed 14:37
\documentclass[11pt]{article}
\usepackage[utf8]{inputenc}
\usepackage[T1]{fontenc}
\usepackage{fixltx2e}
\usepackage{graphicx}
\usepackage{longtable}
\usepackage{float}
\usepackage{wrapfig}
\usepackage{rotating}
\usepackage[normalem]{ulem}
\usepackage{amsmath}
\usepackage{textcomp}
\usepackage{marvosym}
\usepackage{wasysym}
\usepackage{amssymb}
\usepackage{hyperref}
\tolerance=1000
\author{mingzailao}
\date{2016-9-11}
\title{C++ Primer Notes}
\hypersetup{
  pdfkeywords={},
  pdfsubject={},
  pdfcreator={Emacs 24.5.1 (Org mode 8.2.10)}}
\begin{document}

\maketitle
\tableofcontents

\section{Begin}
\label{sec-1}
\subsection{Writing a Simple C++ Program}
\label{sec-1-1}
\subsubsection{Example $1\_1\_1$}
\label{sec-1-1-1}
\begin{verbatim}
#include<iostream>
using std::cout;
using std::endl;
int main()
{
    cout<<"Hello world"<<endl;
    return 0;
}
\end{verbatim}
\subsubsection{Compiling and Executing Our Program}
\label{sec-1-1-2}
\begin{enumerate}
\item Comliling
\label{sec-1-1-2-1}
\begin{verbatim}
#!/bin/bash
cd Code
g++ hello.cpp -o Hello
\end{verbatim}
In the next, I just use Automake to compile.

\item Executing
\label{sec-1-1-2-2}
./Hello
\end{enumerate}
\subsubsection{Exercise}
\label{sec-1-1-3}
\begin{enumerate}
\item Exercise 1.2
\label{sec-1-1-3-1}
Change the program to return -1. A return value of -1 is often treated as an indicator
that the program failed. Recompile and rerun your program to see how your system treats
a failure indicator from main.
\begin{enumerate}
\item Answer
\label{sec-1-1-3-1-1}
\begin{verbatim}
#include<iostream>

int main()
{
    return -1;
}
\end{verbatim}
\end{enumerate}
\end{enumerate}
\subsection{A First Look at Input/Output}
\label{sec-1-2}
\subsubsection{Standard Input and Output Objects}
\label{sec-1-2-1}
The library defines four IO objects:
\begin{enumerate}
\item istream:cin
\item ostream:cout
\item cerr
\item clog
\end{enumerate}
\subsubsection{A Program That Uses the IO Library}
\label{sec-1-2-2}
\begin{verbatim}
#include<iostream>
int main()
{
    std::cout<<"Enter two numbers : "<<std::endl;
    int v1=0,v2=0;
    std::cin>>v1>>v2;
    std::cout<<"The sum of "<<v1<<"and "<<v2<<" is "<<v1+v2<<std::endl;
    return 0;
}
\end{verbatim}
\subsubsection{Writing to a Stream}
\label{sec-1-2-3}
\begin{verbatim}
std::cout<<"Enter two number"<<std::endl;
\end{verbatim}
std::cout is a object in iostream(extends from isteam and ostream), specifically, in 
libary ostream, the operator "<<" 
% Emacs 24.5.1 (Org mode 8.2.10)
\end{document}